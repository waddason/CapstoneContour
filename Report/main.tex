\documentclass[11pt]{article}

% Related to page geomety
\usepackage[a4paper,vmargin={2.4cm,2.8cm},hmargin={2.4cm,2.4cm}]{geometry}
\usepackage[utf8]{inputenc}
\usepackage{fancyhdr}
\pagestyle{fancy}
\renewcommand{\headrulewidth}{1pt}
\fancyhead[C]{CONFIDENTIAL}
\fancyhead[L]{Team 04 - Contour Detection}
\fancyhead[R]{M2DS Capstone projects, IPP, 2025}


% Related to math
\usepackage{amsmath,amssymb,amsfonts,amsthm}

% Related to figures
\usepackage{graphicx}

%Related to referencing
\usepackage{hyperref}
\usepackage{cleveref}

\begin{document}

\begin{center}
\textbf{Contour detection} \\
Fabien Lagnieu, Maha Kadraoui, Tristan Waddington, Abdoul Zeba\\
\textbf{Mentor:} Stéphane Maviel (CEO Vinci/Diane\footnote{Diane is Vinci's inside Startup tasked to find numerical solutions
to ease the work of Vinci collaborators.}) \\\vspace{2em}
\textbf{\Large }
\end{center}
\vspace{-1cm}

Architects are drawing floor plans using a CAD software. These files are used
by engineers to plan and conduct the construction of the building. In this process,
additional measures are needed such as the room size or adjacency. The goal of 
this project is to develop a tool that can automatically detect the contours of 
the rooms in the floor plan and draw it back into their CAD software. 

% This is a latex template for your report. Your report should fit in 5 pages  excluding references and appendices and summarize your main findings and process of thoughts. It should be self-contained, and you should not modify the template. . You can use appendices for additional figures and details, provide explanation of parallel tracks you gave up etc.
\begin{figure}[h]
    \centering
    \includegraphics[width=0.1\textwidth]{figures/Diane.png}
    \hspace{0.1\textwidth}
    \includegraphics[width=0.3\textwidth]{figures/logo_vinci.png}
    \hspace{0.1\textwidth}
    \includegraphics[width=0.1\textwidth]{figures/ipparis.png}

    %\caption{Example figure with example spacing and caption: Left, ipp paris logo. Right, ipp paris logo.}
    \label{fig:my_label}
\end{figure}

\section{Context}
Vinci's engineers are dealing with a different floor plan for every project they are working on.
Every plan has its own drawing style and the contours of the rooms are often mixed with other
lines, but they need this information to plan the location of smoke detectors, fire extinguishers,
and other facilities in the building. The goal of this project is to develop a tool that can
automatically detect the contours of the rooms in the floor plan and draw back the polygons in the CAD software as 
a vector layer for further automatic processing. 
Diane's teams have already worked on this problem and developed a first version of the tool
that is too slow and not accurate enough. They are looking for other approaches
to improve it.
% \textit{What is the question to be answered? Why is it important for the mentor?\\
% Present briefly the dataset (type of data, dimensionality, what else is notable?)}
\paragraph{Datasets}
We were given examples of GeoJson\footnote{The current standard for GeoJson format was 
published in August 2016 by the Internet Engineering Task Force (IETF), 
in \href{https://datatracker.ietf.org/doc/html/rfc7946}{RFC 7946}.} exports of CAD plans. The first dataset was 
composed of 9 GeoJson exports of floor plans :
the geometric description of the drawing lines. 
The second dataset exported the floor plans of 5 whole buildings, including width of walls, and
the contours of the actual rooms as a target.


\section{Methods}
\label{sec:methods}

% \textit{Which methods did you try or could you have tried?\\
% Which one did you choose and why?}\\\\
% Include references if you use a non-standard method \cite{pml1Book}. The corresponding entry is defined in the file \texttt{references.bib}. Do no cite textbooks, \cite{pml1Book} is just an example. 
\label{sec:methods}
After a first phase of data cleaning we have focused on two different paths to solve the problem:

\begin{enumerate}
    \item \textbf{Geometric process} Transform the list of lines into a graph and 
    construct polygons \cite{Schafer2011AutomaticGO} then cluster the lines to 
    create the polygons\cite{dominguez2012Semiautomaticdetection}. This path relies on the
    geometric libraries \texttt{shapely} and \texttt{geopandas}. 
    \item \textbf{Image segmentation} Using a dense literature review\cite{PIZARRO2022104348}, we 
    spotted some successful uses of CNNs\cite{ijgi10020097}. This same paper also provides the list
    of datasets of floor plans as image we can use. Ultimately, this path lead
    to the segmentation of the floor plans into rooms.
\end{enumerate}

\section{Results}
% \textit{Provide convincing plots/tables to support your conclusions.\\
% Justify the metrics you have used to compare methods/assess performance.}

\subsection{The metric to use}
Text of the subsection.

\subsection{The results}
\begin{figure}[h]
    \centering
    \includegraphics[width=0.5\textwidth]{figures/Output6_contours_image.png}

    \caption{Example of the results obtained with image segmentation, each room 
    should be in a different color.}
    \label{fig:result_segmentation}
\end{figure}
\section{Conclusion}
\textit{Summarize and discuss the main findings. What are the limitations?\\
Are the experiments conclusive? With more time, what else would you have tried?}

\paragraph{Organization within the team}
%\textit{How did you share the work among yourselves? (short paragraph, 1/2 page max)}
The project initially started in the office of Vinci/Diane for a brainstorming session.
Since then, we have been working remotely in a smooth way, meeting with 
Vinci/Diane at the beginning of each Thursday afternoon. We were using a shared 
git repository to share our code and results\footnote{Private: https://github.com/waddason/CapstoneContour}.

Tristan took care of the global information management process and organized the
relations with Vinci. After a first quick diverging phase where every one has 
tried different methods and dug into the data,  we have split in teams of two 
to focus on two paths described in \cref{sec:methods}. 
\begin{itemize}
    \item Tristan handled the initial data cleaning, then worked with Maha on 
    the geometric process.
    \item Abdoul explored first the computer vision path and Fabien made the 
    breakthrough in the results with the image segmentation.
\end{itemize}



% Automatic bibliography form references.bib
\bibliographystyle{plain}
\bibliography{references}

\appendix

\newpage
%The appendix does not count towards the 5-page limit. 
\section{Additional details on X}
\label{app:sec:details}

\section{Additional information on X}

\end{document}